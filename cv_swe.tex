%%%%%%%%%%%%%%%%%%%%%%%%%%%%%%%%%%%%%%%%%
% Medium Length Graduate Curriculum Vitae
% LaTeX Templatehttps://www.writelatex.com/1587472bgtrhf#
% Version 1.1 (9/12/12)
%
% This template has been downloaded from:
% http://www.LaTeXTemplates.com
%
% Original author:
% Rensselaer Polytechnic Institute (http://www.rpi.edu/dept/arc/training/latex/resumes/)
%
% Important note:
% This template requires the res.cls file to be in the same directory as the
% .tex file. The res.cls file provides the resume style used for structuring the
% document.
%
%%%%%%%%%%%%%%%%%%%%%%%%%%%%%%%%%%%%%%%%%

%----------------------------------------------------------------------------------------
%	PACKAGES AND OTHER DOCUMENT CONFIGURATIONS
%----------------------------------------------------------------------------------------

\documentclass[margin, 10pt]{res} % Use the res.cls style, the font size can be changed to 11pt or 12pt here

\usepackage{helvet} % Default font is the helvetica postscript font
%\usepackage{newcent} % To change the default font to the new century schoolbook postscript font uncomment this line and comment the one above
\usepackage[T1]{fontenc}
\usepackage[utf8]{inputenc}
\usepackage[swedish]{babel}

\setlength{\textwidth}{5.1in} % Text width of the document

\begin{document}

%----------------------------------------------------------------------------------------
%	NAME AND ADDRESS SECTION
%----------------------------------------------------------------------------------------

\moveleft.5\hoffset\centerline{\large\bf Robin Chowdhury} % Your name at the top
 
\moveleft\hoffset\vbox{\hrule width\resumewidth height 1pt}\smallskip % Horizontal line after name; adjust line thickness by changing the '1pt'
 
\moveleft.5\hoffset\centerline{robinch88@gmail.com} % Your address
\moveleft.5\hoffset\centerline{070-384 41 76}

%----------------------------------------------------------------------------------------

\begin{resume}

%----------------------------------------------------------------------------------------
%	Teknisk kunskap
%----------------------------------------------------------------------------------------

\section{Teknisk kunskap}

{\sl Utvecklingsspråk} \\
Java, Python, Go, Groovy, Matlab, Octave, C, C\#, SQL

{\sl Webb- och mobilutveckling} \\
Javascript, HTML, CSS, Bootstrap, Android, AngularJS, Sails.js, Node.js

{\sl Produkter/Miljöer} \\
Windows XP/Vista/7/8/10, Unix, Solaris, Ubuntu, Mac OS X, Microsoft Office, LibreOffice, Latex, Eclipse, IntelliJ IDEA, Android Studio, Unity, Sublime, VIM, JIRA, Jenkins, MySQL, PostgreSQL

{\sl Versionshanteringssystem} \\
Git, Perforce

%----------------------------------------------------------------------------------------
%	EDUCATION SECTION
%----------------------------------------------------------------------------------------

\section{Utbilding}

{\sl Civilingenjör inom datorteknik (300 hp)} \\
Kungliga Tekniska Högskolan \\
Inriktar mig inom datalogi för att få en bred bas inom mjukvaruutveckling.

 
%----------------------------------------------------------------------------------------
%	PROFESSIONAL EXPERIENCE SECTION
%----------------------------------------------------------------------------------------
 
\section{Arbetslivs-\\erfarenhet}

{\sl Full-stack utvecklare} \hfill Sommar 2013 \\
Scania, Research and Development

\begin{itemize} \itemsep -2pt % Reduce space between items
\item Visualiserade information från JIRA, som är ett populärt ärendehanteringssystem.
\end{itemize}	
 
{\sl Android utvecklare} \hfill Sommar 2012 \\
Scania, Research and Development
\begin{itemize} 
\item Utvecklade en app till surfplattor som läser av information från Scanias lastbilar i realtid och presenterar dem grafiskt.
\end{itemize} 

{\sl Övningsassistent} \hfill Våren 2012 \\
KTH, Databasteknik

{\sl Laborationsassistent} \hfill Våren 2012 \\
KTH, Databasteknik

{\sl Expedit} \hfill Somrarna 2008-2011 \\
Roslagsjärn med färg AB

%----------------------------------------------------------------------------------------
%	Projekt
%---------------------------------------------------------------------------------------- 

\section{Projekt}

{\sl Context Expansion using Random Indexing} \hfill 2015 \\
KTH, Sökmotorer och informationssökningssystem
\begin{itemize} 
\item Utvecklade en synonymordlista som använde Apache Solr, vilket är en populär sökplattform byggd på öppen källkod, för att utvidga sökfrågor. Synonymordlistan skapades genom att använda metoden Random Indexing på data från Wikipedia.
\end{itemize} 

\begin{minipage}{\textwidth}
{\sl Visualizing Wikipedia using a Graph Database} \hfill 2015 \\
KTH, Moderna databassystem och databastillämpningar
\begin{itemize} 
\item Visualiserade Wikipedia genom att spara dess länkstruktur i grafdatabasen Neo4j och visualisera detta med olika klusteralgoritmer.
\end{itemize} 
\end{minipage}

{\sl Re:Peter} \hfill 2014 \\
KTH, Datorspelsdesign
\begin{itemize} 
\item Ett pussel- och äventyrsspel som är kompatibelt med Oculus Rift och Xbox-kontroller. Spelet utvecklades i Unity.
\end{itemize} 

{\sl Studs} \hfill 2013-2014 \\
KTH, Datasektionen
\begin{itemize} 
\item Vi var 31 studenter i årskurs 4-5 som skapade event tillsammans med diverse företag för att lära känna dem. Via sponsring från dessa företag lyckades vi även åka till New York, San Francisco och Los Angeles för att nätverka med ledande företag så som Facebook, Google, Yelp och Palantir.
\end{itemize} 

{\sl EyeBrowse} \hfill 2009-2010 \\
KTH, Kandidat
\begin{itemize} 
\item Vi var tio personer som utvecklade en ögonstyrd browser. Vi använde Tobiis API:er och hårdvara.
\end{itemize} 

{\sl Estimating Complex Blur Patterns in Large Real-life Images} \hfill 2012 \\
KTH, Kandidatuppsats \\
Skriven med Alexander Solsmed.


%----------------------------------------------------------------------------------------
%	EXTRA-CURRICULAR ACTIVITIES SECTION
%----------------------------------------------------------------------------------------

\section{Övriga meriter}

{\sl Näringslivsgruppen} \hfill 2012-2014 \\
KTH, Datasektionen


{\sl Företagsvärd för Google} \hfill 2011 \\
KTH, Armada

{\sl Ekonomimästare, Datas klubbmästeri (DKM)} \hfill 2011-2012 \\
KTH, Datasektionen

{\sl Datas klubbmästeri (DKM)} \hfill 2010-2011 \\
KTH, Datasektionen
\begin{itemize} 
\item Arrangerar pubar, event och fester för datastudenter.
\end{itemize} 

{\sl Mottagningen} \hfill 2010, 2011 \\
KTH, Datasektionen
\begin{itemize} 
\item Arrangerar mottagningen för nyantagna datastudenter.
\end{itemize} 


%----------------------------------------------------------------------------------------
%	Språk
%----------------------------------------------------------------------------------------

\section{Språk}
Svenska, modersmål \\
Engelska, flytande i tal och skrift

%----------------------------------------------------------------------------------------
%	Referenser
%----------------------------------------------------------------------------------------

\section{Referenser}
Referenser lämnas på förfrågan.

%----------------------------------------------------------------------------------------

\end{resume}
\end{document}