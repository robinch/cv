%%%%%%%%%%%%%%%%%%%%%%%%%%%%%%%%%%%%%%%%%
% Medium Length Graduate Curriculum Vitae
% LaTeX Templatehttps://www.writelatex.com/1587472bgtrhf#
% Version 1.1 (9/12/12)
%
% This template has been downloaded from:
% http://www.LaTeXTemplates.com
%
% Original author:
% Rensselaer Polytechnic Institute (http://www.rpi.edu/dept/arc/training/latex/resumes/)
%
% Important note:
% This template requires the res.cls file to be in the same directory as the
% .tex file. The res.cls file provides the resume style used for structuring the
% document.
%
%%%%%%%%%%%%%%%%%%%%%%%%%%%%%%%%%%%%%%%%%

%----------------------------------------------------------------------------------------
%	PACKAGES AND OTHER DOCUMENT CONFIGURATIONS
%----------------------------------------------------------------------------------------

\documentclass[margin, 10pt]{res} % Use the res.cls style, the font size can be changed to 11pt or 12pt here

\usepackage{helvet} % Default font is the helvetica postscript font
%\usepackage{newcent} % To change the default font to the new century schoolbook postscript font uncomment this line and comment the one above
\usepackage[T1]{fontenc}
\usepackage[utf8]{inputenc}
\usepackage[swedish]{babel}

\setlength{\textwidth}{5.1in} % Text width of the document

\begin{document}

%----------------------------------------------------------------------------------------
%	NAME AND ADDRESS SECTION
%----------------------------------------------------------------------------------------

\moveleft.5\hoffset\centerline{\large\bf Robin Chowdhury} % Your name at the top
 
\moveleft\hoffset\vbox{\hrule width\resumewidth height 1pt}\smallskip % Horizontal line after name; adjust line thickness by changing the '1pt'
 
\moveleft.5\hoffset\centerline{robinch88@gmail.com} % Your address
\moveleft.5\hoffset\centerline{070-384 41 76}

%----------------------------------------------------------------------------------------

\begin{resume}

%----------------------------------------------------------------------------------------
%	EDUCATION SECTION
%----------------------------------------------------------------------------------------

\section{Utbilding}

{\sl Civilingengör i datorteknik (300hp)} \\
Kungliga Tekniska Högskolan \\
Inriktar mig inom datalogi för att få en bred bas inom mjukvaruutveckling.

 
%----------------------------------------------------------------------------------------
%	PROFESSIONAL EXPERIENCE SECTION
%----------------------------------------------------------------------------------------
 
\section{Arbetslivs-\\erfarenhet}

{\sl Mjukvaruutvecklare} \hfill Sommar 2013 \\
Scania, Research and development

\begin{itemize} \itemsep -2pt % Reduce space between items
\item Visualiserar parsad information från JIRA
\end{itemize}	
 
{\sl Andriod-utvecklare} \hfill Sommar 2012 \\
Scania, Research and development
\begin{itemize} 
\item Utvecklade en app till surfplattor som läser av information från Scanias lastbilar i real-time och presenterar de grafiskt.
\end{itemize} 

{\sl Övningsassistent} \hfill Våren 2012 \\
KTH, Databasteknik (DD1334)

{\sl Laborationsassistent} \hfill Våren 2012 \\
KTH, Databasteknik (DD1334)

{\sl Expedit} \hfill Sommrarna 2008-2011 \\
Roslagsjärn med färg AB

%----------------------------------------------------------------------------------------
%	Projekt
%---------------------------------------------------------------------------------------- 

\section{Projekt}

{\sl Studs} \hfill 2013-2014 \\
KTH, Datasektionen
\begin{itemize} 
\item Vi var 31 studenter i årskurs 4-5 som skapade event tillsammans med diverse företag för att lära känna dem. Med pengarna som vi samlade under dessa event åkte vill till New York, San Francisco och Los Angeles för att träffa fler företag för skapa ett starkare band mellan datasektionen och näringslivet.
\end{itemize} 

{\sl EyeBorwse} \hfill 2009-2010 \\
KTH, Kandidat
\begin{itemize} 
\item Vi var tio personer som utvecklade en ögonstyrd browser. Vi använde Tobiis API och hårdvara.
\end{itemize} 

{\sl Estimating complex blur patterns in large real-life images} \hfill 2012 \\
KTH, Kandidatuppsats \\
Skriven med Alexander Solsmed


%----------------------------------------------------------------------------------------
%	EXTRA-CURRICULAR ACTIVITIES SECTION
%----------------------------------------------------------------------------------------

\section{Övriga meriter}

{\sl Näringslivsgruppen} \hfill 2012-2014 \\
KTH, Datasektionen


{\sl Företagsvärd för Google} \hfill 2011 \\
KTH, Armada

{\sl Ekonomimästare, Datas klubbmästeri (DKM)} \hfill 2011-2012 \\
KTH, Datasektionen

{\sl Datas klubbmästeri (DKM)} \hfill Våren 2010-2011 \\
KTH, Datasektionen
\begin{itemize} 
\item Arrangerar pubar, event och fester för datastudenter.
\end{itemize} 

{\sl Mottagningen} \hfill Våren 2010, 2011 \\
KTH, Datasektionen
\begin{itemize} 
\item Arrangerar mottagningen för nyantagna datastudenter.
\end{itemize} 

%----------------------------------------------------------------------------------------
%	Språk
%----------------------------------------------------------------------------------------

\section{Språk}
Svenska, modersmål \\
Engelska, flytande i tal och skrift

%----------------------------------------------------------------------------------------
%	Kompetens
%----------------------------------------------------------------------------------------

\section{Kompetens}
{\sl Programmering} \\
Java, Android, Python, Go, Javascript, SQL

{\sl Mjukvara} \\
Eclipse, Windows, MS Office, Sublime

%----------------------------------------------------------------------------------------
%	Referenser
%----------------------------------------------------------------------------------------

\section{Referenser}
Referenser lämnas på förfrågan

%----------------------------------------------------------------------------------------

\end{resume}
\end{document}