%%%%%%%%%%%%%%%%%%%%%%%%%%%%%%%%%%%%%%%%%
% Medium Length Graduate Curriculum Vitae
% LaTeX Templatehttps://www.writelatex.com/1587472bgtrhf#
% Version 1.1 (9/12/12)
%
% This template has been downloaded from:
% http://www.LaTeXTemplates.com
%
% Original author:
% Rensselaer Polytechnic Institute (http://www.rpi.edu/dept/arc/training/latex/resumes/)
%
% Important note:
% This template requires the res.cls file to be in the same directory as the
% .tex file. The res.cls file provides the resume style used for structuring the
% document.
%
%%%%%%%%%%%%%%%%%%%%%%%%%%%%%%%%%%%%%%%%%

%----------------------------------------------------------------------------------------
%	PACKAGES AND OTHER DOCUMENT CONFIGURATIONS
%----------------------------------------------------------------------------------------

\documentclass[margin, 10pt]{res} % Use the res.cls style, the font size can be changed to 11pt or 12pt here

\usepackage{helvet} % Default font is the helvetica postscript font
%\usepackage{newcent} % To change the default font to the new century schoolbook postscript font uncomment this line and comment the one above
\usepackage[T1]{fontenc}
\usepackage[utf8]{inputenc}
\usepackage[english]{babel}

\setlength{\textwidth}{5.1in} % Text width of the document

\begin{document}

%----------------------------------------------------------------------------------------
%	NAME AND ADDRESS SECTION
%----------------------------------------------------------------------------------------

\moveleft.5\hoffset\centerline{\large\bf Robin Chowdhury} % Your name at the top
 
\moveleft\hoffset\vbox{\hrule width\resumewidth height 1pt}\smallskip % Horizontal line after name; adjust line thickness by changing the '1pt'
 
\moveleft.5\hoffset\centerline{robinch88@gmail.com} % Your address
\moveleft.5\hoffset\centerline{070-384 41 76}

%----------------------------------------------------------------------------------------

\begin{resume}

%----------------------------------------------------------------------------------------
%	EDUCATION SECTION
%----------------------------------------------------------------------------------------

\section{Education}

{\sl Master's Degree Computer Science (300hp)} \\
Kungliga Tekniska Högskolan \\
(Royal Institute of Technology, Sweden)

 
%----------------------------------------------------------------------------------------
%	PROFESSIONAL EXPERIENCE SECTION
%----------------------------------------------------------------------------------------
 
\section{Experience}

{\sl Software Developer} \hfill Summer 2013 \\
Scania, Research and Development

\begin{itemize} \itemsep -2pt % Reduce space between items
\item Visualize parsed information from JIRA
\end{itemize}	
 
{\sl Andriod developer} \hfill Summer 2012 \\
Scania, Research and Development
\begin{itemize} 
\item Visualize real-time data from Scania trucks.
\end{itemize} 

{\sl Teaching assistant} \hfill Spring 2012 \\
KTH, Database technology (DD1334)

{\sl Lab assistant} \hfill Spring 2012 \\
KTH, Database technology (DD1334)

{\sl Salseman} \hfill Summers 2008-2011 \\
Roslagsjärn med färg AB

%----------------------------------------------------------------------------------------
%	Projekt
%---------------------------------------------------------------------------------------- 

\section{Projects}

{\sl Studs} \hfill 2013-2014 \\
KTH, Datasektionen
\begin{itemize} 
\item We were 31 students that created events together with various companies to meet and get to know them. With the money collected from these events, we managed to travel to New York, San Francisco and Los Angeles to meet other companies.
\end{itemize} 

{\sl EyeBorwse} \hfill 2009-2010 \\
KTH, Bachelor
\begin{itemize} 
\item Developed an eye-controlled browser with nine other students. We used Tobii’s API and hardware.
\end{itemize} 

{\sl Estimating complex blur patterns in large real-life images} \hfill 2012 \\
KTH, Bachelor Thesis \\
Written with Alexander Solssmed


%----------------------------------------------------------------------------------------
%	EXTRA-CURRICULAR ACTIVITIES SECTION
%----------------------------------------------------------------------------------------

\section{Extra-curricular activities}

{\sl Business Relations Club} \hfill 2012-2014 \\
KTH, Datasektionen
\begin{itemize} 
\item Facilitated relations between students and companies by arranging events and lectures. One event is when over 30 companies comes to KTH for a day to meet and greet students.
\end{itemize}

{\sl Business Host for Google} \hfill 2011 \\
KTH, Armada

{\sl Financial Manager, Datas klubbmästeri (DKM)} \hfill 2011-2012 \\
KTH, Datasektionen

{\sl Datas klubbmästeri (DKM)} \hfill 2010-2011 \\
KTH, Datasektionen
\begin{itemize} 
\item Arrange pubs, events and parties for Computer Science students.
\end{itemize}

{\sl Orientation} \hfill 2010, 2011 \\
KTH, Datasektionen
\begin{itemize} 
\item Helped with organizing the orientation of new Computer Science students.
\end{itemize}

%----------------------------------------------------------------------------------------
%	Språk
%----------------------------------------------------------------------------------------

\section{Languages}
Swedish, mother tongue \\
English, fluent

%----------------------------------------------------------------------------------------
%	Kompetens
%----------------------------------------------------------------------------------------

\section{Competense}
{\sl Programming} \\
Java, Android, Python, Go, Javascript, SQL, CSS, HTML

{\sl Software} \\
Eclipse, Windows, Sublime, Ubuntu

%----------------------------------------------------------------------------------------

\end{resume}
\end{document}