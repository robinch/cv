%%%%%%%%%%%%%%%%%%%%%%%%%%%%%%%%%%%%%%%%%
% Medium Length Graduate Curriculum Vitae
% LaTeX Templatehttps://www.writelatex.com/1587472bgtrhf#
% Version 1.1 (9/12/12)
%
% This template has been downloaded from:
% http://www.LaTeXTemplates.com
%
% Original author:
% Rensselaer Polytechnic Institute (http://www.rpi.edu/dept/arc/training/latex/resumes/)
%
% Important note:
% This template requires the res.cls file to be in the same directory as the
% .tex file. The res.cls file provides the resume style used for structuring the
% document.
%
%%%%%%%%%%%%%%%%%%%%%%%%%%%%%%%%%%%%%%%%%

%----------------------------------------------------------------------------------------
%	PACKAGES AND OTHER DOCUMENT CONFIGURATIONS
%----------------------------------------------------------------------------------------

\documentclass[margin, 10pt]{res} % Use the res.cls style, the font size can be changed to 11pt or 12pt here

\usepackage{helvet} % Default font is the helvetica postscript font
%\usepackage{newcent} % To change the default font to the new century schoolbook postscript font uncomment this line and comment the one above
\usepackage[T1]{fontenc}
\usepackage[utf8]{inputenc}
\usepackage[english]{babel}

\setlength{\textwidth}{5.1in} % Text width of the document

\begin{document}

%----------------------------------------------------------------------------------------
%	NAME AND ADDRESS SECTION
%----------------------------------------------------------------------------------------

\moveleft.5\hoffset\centerline{\large\bf Robin Chowdhury} % Your name at the top
 
\moveleft\hoffset\vbox{\hrule width\resumewidth height 1pt}\smallskip % Horizontal line after name; adjust line thickness by changing the '1pt'
 
\moveleft.5\hoffset\centerline{robinch88@gmail.com} % Your address
\moveleft.5\hoffset\centerline{+46 (0)70 384 41 76}

%----------------------------------------------------------------------------------------

\begin{resume}

%----------------------------------------------------------------------------------------
%	Technical Knowledge
%----------------------------------------------------------------------------------------

\section{Technical Knowledge}

{\sl Programming Languages} \\
Java, C\#, Python, Go, Groovy, C, SQL

{\sl Web and Mobile Development} \\
Android, Xamarin, Javascript, HTML, CSS, AngularJS, Sails.js, Node.js

{\sl Products/Environments} \\
Android Studio, Xamarin Studio, Visual Studio, Eclipse, IntelliJ IDEA, JIRA, MySQL, PostgreSQL

{\sl Version Control Systems} \\
Git, Perforce


%----------------------------------------------------------------------------------------
%	PROFESSIONAL EXPERIENCE SECTION
%----------------------------------------------------------------------------------------
 
\section{Experience}

{\sl Forward Deployed Android Engineer} \hfill April 2016 - Present \\
Tink, ABN AMRO, Amsterdam
\begin{itemize} \itemsep -2pt % Reduce space between items
\item Responsible for developing and releasing the Android version of Grip. A personal finance manager based the Tink app.
\end{itemize}

{\sl Full-stack Developer} \hfill Summer 2013 \\
Scania, Research and Development
\begin{itemize} \itemsep -2pt % Reduce space between items
\item Visualized information from JIRA
\end{itemize}	
 
{\sl Android Developer} \hfill Summer 2012 \\
Scania, Research and Development
\begin{itemize} 
\item Visualized real-time data from Scania trucks.
\end{itemize} 

{\sl Teaching Assistant} \hfill Spring 2012 \\
KTH, Database technology (DD1334)

{\sl Lab Assistant} \hfill Spring 2012 \\
KTH, Database technology (DD1334)

{\sl Salesman} \hfill Summers 2008-2011 \\
Roslagsjärn med färg AB


%----------------------------------------------------------------------------------------
%	EDUCATION SECTION
%----------------------------------------------------------------------------------------

\section{Education}

{\sl Master's Degree Computer Science (300 hp)} \\
Kungliga Tekniska Högskolan \\
(Royal Institute of Technology, Sweden)

 
%----------------------------------------------------------------------------------------
%	Projekt
%---------------------------------------------------------------------------------------- 

\section{Projects}

{\sl Context Expansion using Random Indexing} \hfill 2015 \\
KTH, Search Engines and Information Retrieval Systems
\begin{itemize} 
\item Developed a thesaurus that was using Apache Solr, a popular open source search platform, to expand search queries. The thesaurus was created by using the method Random Indexing on data from Wikipedia.
\end{itemize} 

{\sl Visualizing Wikipedia using a Graph Database} \hfill 2015 \\
KTH, Modern Database Systems and Their Applications
\begin{itemize} 
\item Visualized Wikipedia by storing its link structure in the graph database Neo4j and using different clustering algorithms to visualize.
\end{itemize} 

{\sl Re:Peter} \hfill 2014 \\
KTH, Computer Game Design
\begin{itemize} 
\item A puzzle-adventure game that is compatible with Oculus Rift and Xbox controller. The game was developed in Unity.
\end{itemize} 

{\sl Studs} \hfill 2013-2014 \\
KTH, Datasektionen
\begin{itemize} 
\item We were 31 students that created events together with various companies to meet and get to know them. With the sponsorship collected from the companies we were able to travel to New York, San Francisco and Los Angeles where we networked with leading companies such as Facebook, Google, Yelp and Palantir.
\end{itemize} 

{\sl EyeBrowse} \hfill 2009-2010 \\
KTH, Bachelor
\begin{itemize} 
\item Developed an eye-controlled browser with nine other students. We used Tobii’s APIs and hardware.
\end{itemize} 

{\sl Estimating Complex Blur Patterns in Large Real-life Images} \hfill 2012 \\
KTH, Bachelor Thesis \\
Written with Alexander Solsmed.


%----------------------------------------------------------------------------------------
%	EXTRA-CURRICULAR ACTIVITIES SECTION
%----------------------------------------------------------------------------------------

\section{Extra-curricular Activities}

{\sl Business Relations Club} \hfill 2012-2014 \\
KTH, Datasektionen
\begin{itemize} 
\item Facilitated relations between students and companies by arranging events and lectures. One event is when over 30 companies comes to KTH for a day to network with students.
\end{itemize}

{\sl Business Host for Google} \hfill 2011 \\
KTH, Armada

{\sl Financial Manager, Datas klubbmästeri (DKM)} \hfill 2011-2012 \\
KTH, Datasektionen

{\sl Datas klubbmästeri (DKM)} \hfill 2010-2011 \\
KTH, Datasektionen
\begin{itemize} 
\item Arranged pubs, events and parties for Computer Science students.
\end{itemize}

{\sl Orientation} \hfill 2010, 2011 \\
KTH, Datasektionen
\begin{itemize} 
\item Helped with organizing the orientation of new Computer Science students.
\end{itemize}

%----------------------------------------------------------------------------------------
%	Språk
%----------------------------------------------------------------------------------------

\section{Languages}
Swedish, native language \\
English, fluent in speech and writing

%----------------------------------------------------------------------------------------

\end{resume}
\end{document}
